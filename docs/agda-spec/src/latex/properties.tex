\newcommand{\Val}{\fun{Val}}
\newcommand{\POV}[1]{\ensuremath{\mathsf{PresOfVal}(\mathsf{#1})}}
\newcommand{\DBE}[2]{\ensuremath{\mathsf{DBE}({#1},~{#2})}}
\newcommand{\DGO}[2]{\ensuremath{\mathsf{DGO}({#1},~{#2})}}
\newcommand{\transtar}[2]{\xlongrightarrow[\textsc{#1}]{#2}\negthickspace^{*}}
\newcommand{\StabilityWindow}{\ensuremath{\mathsf{StabilityWindow}}}

\section{Properties}
\label{sec:properties}

This section describes the properties that the consensus layer should have.
The goal is to include these properties in the executable specification
to enable e.g. property-based testing or formal verification.

\subsection{Header-Only Validation}
\label{sec:header-only-validation}
In any given chain state, the consensus layer needs to be able to validate the
block headers without having to download the block bodies.
Property~\ref{prop:header-only-validation} states that if an extension of a
chain that spans less than $\StabilityWindow$ slots is valid, then validating the
headers of that extension is also valid. This property is useful for its
converse: if the header validation check for a sequence of headers does not
pass, then we know that the block validation that corresponds to those headers
will not pass either. In these properties, we refer to the $\mathsf{CHAIN}$
transition system as defined in \cite{shelley_chain_spec}.

\begin{property}[Header only validation]\label{prop:header-only-validation}
  For all states $s$ with slot number $t$\footnote{i.e. the
    component $\var{s_\ell}$ of the last applied block of $s$ equals $t$},
    and chain extensions $E$ with corresponding headers $H$ such that:
  %
  $$
  0 \leq t_E - t  \leq \StabilityWindow
  $$
  %
  we have:
  %
  $$
  \vdash s \transtar{chain}{E} s'
  \implies
  \var{nes} \vdash \tilde{s} \transtar{\hyperref[fig:ts-rules:chainhead]{chainhead}}{H} \tilde{s}'
  $$
  where $s=(\var{nes},~\tilde{s})$,
  $t_E$ is the maximum slot number appearing in the blocks contained in
  $E$, and $H$ is obtained from $E$ by extracting the header from each block in $E$.
\end{property}

\begin{property}[Body only validation]\label{prop:body-only-validation}
  For all states $s$ with slot number $t$, and chain
  extensions $E = [b_0, \ldots, b_n]$ with corresponding headers $H = [h_0, \ldots, h_n]$ such that:
  $$
  0 \leq t_E - t  \leq \StabilityWindow
  $$
  we have that for all $i \in [1, n]$:
  $$
  \var{nes} \vdash \tilde{s} \transtar{\hyperref[fig:ts-rules:chainhead]{chainhead}}{H} s_{h}
  \wedge
  \vdash (\var{nes},~\tilde{s}) \transtar{chain}{[b_0, \ldots, b_{i-1}]} s_{i-1}
  \implies
  \var{nes'} \vdash \tilde{s}_{i-1}\trans{\hyperref[fig:ts-rules:chainhead]{chainhead}}{h_i} s'_{h}
  $$
  where $s=(\var{nes},~\tilde{s})$, $s_{i-1}=(\var{nes'},~\tilde{s}_{i-1})$,
  $t_E$ is the maximum slot number appearing in the blocks contained in $E$.
\end{property}

Property~\ref{prop:body-only-validation} states that if we validate a sequence
of headers, we can validate their bodies independently and be sure that the
blocks will pass the chain validation rule. To see this, given an environment
$e$ and initial state $s$, assume that a sequence of headers
$H = [h_0, \ldots, h_n]$ corresponding to blocks in $E = [b_0, \ldots, b_n]$ is
valid according to the $\mathsf{CHAINHEAD}$ transition system:
%
$$
\var{nes} \vdash \tilde{s} \transtar{\hyperref[fig:ts-rules:chainhead]{chainhead}}{H} \tilde{s}'
$$
%
Assume the bodies of $E$ are valid
according to the $\mathsf{BBODY}$ rules (defined in \cite{shelley_chain_spec}), but $E$ is not valid according to
the $\mathsf{CHAIN}$ rule. Assume that there is a $b_j \in E$ such that it is
\textbf{the first block} such that does not pass the $\mathsf{CHAIN}$
validation. Then:
%
$$
\vdash (\var{nes},~\tilde{s}) \transtar{chain}{[b_0, \ldots, b_{j-1}]} s_j
$$
But by Property~\ref{prop:body-only-validation} we know that
%
$$
\var{nes}_j \vdash \tilde{s}_j \trans{\hyperref[fig:ts-rules:chainhead]{chainhead}}{h_j} \tilde{s}_{j+1}
$$
which means that block $b_j$ has valid headers, and this in turn means that the
validation of $b_j$ according to the chain rules must have failed because it
contained an invalid block body. But this contradicts our assumption that the
block bodies were valid.

\clearpage
