\section{Blockchain Layer}
\label{sec:chain}

\newcommand{\BlocksMade}{\type{BlocksMade}}
\newcommand{\XOR}{\mathsf{XOR}}

This chapter introduces the view of the blockchain layer as required for the
consensus. The main transition rule is $\mathsf{CHAINHEAD}$ which calls the subrules
$\mathsf{TICKF}$, $\mathsf{TICKN}$ and $\mathsf{PRTCL}$.

\subsection{Block Definitions}
\label{sec:defs-blocks}

\begin{figure*}[htb]
  \emph{Abstract types}
  %
  \begin{equation*}
    \begin{array}{r@{~\in~}lr}
      \var{h} & \HashHeader & \text{hash of a block header}\\
      \var{hb} & \HashBBody & \text{hash of a block body}\\
      \var{bn} & \BlockNo & \text{block number}\\
      \eta & \Nonce & \text{nonce}\\
      \var{vrfRes} & \VRFRes & \text{VRF result value}\\
    \end{array}
  \end{equation*}
  %
  \emph{Operational Certificate}
  %
  \begin{equation*}
    \OCert =
    \left(
      \begin{array}{r@{~\in~}lr}
        \var{vk_{hot}} & \VKeyEv & \text{operational (hot) key}\\
        \var{n} & \N & \text{certificate issue number}\\
        c_0 & \KESPeriod & \text{start KES period}\\
        \sigma & \Sig & \text{cold key signature}\\
      \end{array}
    \right)
  \end{equation*}
  %
  \emph{Block Header Body}
  %
  \begin{equation*}
    \BHBody =
    \left(
      \begin{array}{r@{~\in~}lr}
        \var{prevHeader} & \HashHeader^? & \text{hash of previous block header}\\
        \var{issuerVk} & \VKey & \text{block issuer}\\
        \var{vrfVk} & \VKey & \text{VRF verification key}\\
        \var{blockNo} & \BlockNo & \text{block number}\\
        \var{slot} & \Slot & \text{block slot}\\
        \var{vrfRes} & \VRFRes & \text{VRF result value}\\
        \var{vrfPrf} & \Proof & \text{VRF proof}\\
        \var{bodySize} & \N & \text{size of the block body}\\
        \var{bodyHash} & \HashBBody & \text{block body hash}\\
        \var{oc} & \OCert & \text{operational certificate}\\
        \var{pv} & \ProtVer & \text{protocol version}\\
      \end{array}
    \right)
  \end{equation*}
  %
  \emph{Block Types}
  %
  \begin{equation*}
    \begin{array}{r@{~\in~}l@{\qquad=\qquad}l}
      \var{bh}
      & \BHeader
      & \BHBody \times \Sig
    \end{array}
  \end{equation*}
  \emph{Abstract functions}
  %
  \begin{equation*}
    \begin{array}{r@{~\in~}lr}
      \nonceOp{} & \Nonce \to \Nonce \to \Nonce
                 & \text{binary nonce operation} \\
      \headerHash{} & \BHeader \to \HashHeader
                   & \text{hash of a block header} \\
      \headerSize{} & \BHeader \to \N
                   & \text{size of a block header} \\
      \slotToSeed{} & \Slot \to \Seed
                    & \text{convert a slot to a seed} \\
      \nonceToSeed{} & \Nonce \to \Seed
                     & \text{convert a nonce to a seed} \\
      \prevHashToNonce{} & \HashHeader^? \to \Seed
                    & \text{convert an optional header hash to a seed} \\
    \end{array}
  \end{equation*}
  %
  \emph{Accessor Functions}
  \begin{equation*}
    \begin{array}{r@{~\in~}l@{~~~~~~~~~~}r@{~\in~}lr}
      \fun{blockHeader} & \Block \to \BHeader &
      \fun{headerBody} & \BHeader \to \BHBody \\
      \fun{headerSig} & \BHeader \to \Sig &
      \fun{hBVkCold} & \BHBody \to \VKey \\
      \fun{hBVrfVk} & \BHBody \to \VKey &
      \fun{hBPrevHeader} & \BHBody \to \HashHeader^? \\
      \fun{hBSlot} & \BHBody \to \Slot &
      \fun{hBBlockNo} & \BHBody \to \BlockNo \\
      \fun{hBVrfRes} & \BHBody \to \VRFRes &
      \fun{hBVrfPrf} & \BHBody \to \Proof \\
      \fun{hBBodySize} & \BHBody \to \N &
      \fun{hBBodyHash} & \BHBody \to \HashBBody \\
      \fun{hBOCert} & \BHBody \to \OCert &
      \fun{hBProtVer} & \BHBody \to \ProtVer \\
    \end{array}
  \end{equation*}
  %
  \caption{Block Definitions}
  \label{fig:defs:blocks}
\end{figure*}

%%
%% Figure - Helper Functions for Blocks
%%
\begin{figure}[htb]
  \emph{Block Helper Functions}
  \begin{align*}
      & \fun{hBLeader} \in \BHBody \to \N \\
      & \fun{hBLeader}~\var{bhb} = \fun{hash}~(``L"~|~(\fun{hBVrfRes}~\var{bhb}))\\
      \\
      & \fun{hBNonce} \in \BHBody \to \Nonce \\
      & \fun{hBNonce}~\var{bhb} = \fun{hash}~(``N"~|~(\fun{hBVrfRes}~\var{bhb}))\\
  \end{align*}
  %
  \caption{Helper Functions used for Blocks}
  \label{fig:funcs:blocks-helper}
\end{figure}

\clearpage

\subsection{$\mathsf{TICKN}$ Transition}
\label{sec:tick-nonce-trans}

The Tick Nonce Transition ($\mathsf{TICKN}$) is responsible for updating the epoch nonce and the
previous epoch's hash nonce at the start of an epoch. Its environment is shown in
Figure~\ref{fig:ts-types:ticknonce} and consists of the candidate nonce $\eta_c$
and the previous epoch's last block header hash as a nonce $\eta_\var{ph}$.
Its state consists of the epoch nonce $\eta_0$ and the previous epoch's last block header hash nonce $\eta_h$.

\begin{figure}
  \emph{Tick Nonce environments}
  \begin{equation*}
    \TickNonceEnv =
    \left(
      \begin{array}{r@{~\in~}lr}
        \eta_c & \Nonce & \text{candidate nonce} \\
        \eta_\var{ph} & \Nonce & \text{previous header hash as nonce} \\
      \end{array}
    \right)
  \end{equation*}
  %
  \emph{Tick Nonce states}
  \begin{equation*}
    \TickNonceState =
    \left(
      \begin{array}{r@{~\in~}lr}
        \eta_0 & \Nonce & \text{epoch nonce} \\
        \eta_h & \Nonce & \text{nonce from hash of previous epoch's last block header} \\
      \end{array}
    \right)
  \end{equation*}

  \caption{Tick Nonce types}
  \label{fig:ts-types:ticknonce}
\end{figure}

The signal to the transition rule $\mathsf{TICKN}$ is a marker indicating
whether we are in a new epoch. If we are in a new epoch, we update the epoch
nonce and the previous hash. Otherwise, we do nothing.

\begin{figure}[ht]
  \begin{equation}\label{eq:tick-nonce-notnewepoch}
   \inference[Not-New-Epoch]
   { }
   {
     {\begin{array}{c}
        \eta_c \\
        \eta_\var{ph} \\
      \end{array}}
     \vdash
     {\left(\begin{array}{c}
           \eta_0 \\
           \eta_h \\
     \end{array}\right)}
     \trans{tickn}{\mathsf{False}}
     {\left(\begin{array}{c}
           \eta_0 \\
           \eta_h \\
     \end{array}\right)}
   }
 \end{equation}

 \nextdef

 \begin{equation}\label{eq:tick-nonce-newepoch}
   \inference[New-Epoch]
   {}
   {
     {\begin{array}{c}
        \eta_c \\
        \eta_\var{ph} \\
      \end{array}}
     \vdash
     {\left(\begin{array}{c}
           \eta_0 \\
           \eta_h \\
     \end{array}\right)}
     \trans{tickn}{\mathsf{True}}
     {\left(\begin{array}{c}
           \varUpdate{\eta_c \nonceOp \eta_h} \\
           \varUpdate{\eta_\var{ph}} \\
     \end{array}\right)}
   }
 \end{equation}

 \caption{Tick Nonce rules}
 \label{fig:rules:tick-nonce}
\end{figure}

\subsection{$\mathsf{UPDN}$ Transition}
\label{sec:update-nonces-trans}

The Update Nonce Transition ($\mathsf{UPDN}$) updates the nonces until the randomness gets fixed.
The environment is shown in Figure~\ref{fig:ts-types:updnonce} and consists of
the block nonce $\eta$.
The state is shown in Figure~\ref{fig:ts-types:updnonce} and consists of
the candidate nonce $\eta_c$ and the evolving nonce $\eta_v$.

\begin{figure}
  \emph{Update Nonce environments}
  \begin{equation*}
    \UpdateNonceEnv =
    \left(
      \begin{array}{r@{~\in~}lr}
        \eta & \Nonce & \text{new nonce} \\
      \end{array}
    \right)
  \end{equation*}
  %
  \emph{Update Nonce states}
  \begin{equation*}
    \UpdateNonceState =
    \left(
      \begin{array}{r@{~\in~}lr}
        \eta_v & \Nonce & \text{evolving nonce} \\
        \eta_c & \Nonce & \text{candidate nonce} \\
      \end{array}
    \right)
  \end{equation*}
  %
  \emph{Update Nonce Transitions}
  \begin{equation*}
    \_ \vdash \var{\_} \trans{updn}{\_} \var{\_} \subseteq
    \powerset (\UpdateNonceEnv
               \times \UpdateNonceState
               \times \Slot
               \times \UpdateNonceState
              )
  \end{equation*}
  \caption{Update Nonce transition-system types}
  \label{fig:ts-types:updnonce}
\end{figure}

The transition rule $\mathsf{UPDN}$ takes the slot \var{s} as signal. There are
two different cases for $\mathsf{UPDN}$: one where \var{s} is not yet
\RandomnessStabilisationWindow{}\footnote{Note that in pre-Conway eras \StabilityWindow{}
was used instead of \RandomnessStabilisationWindow{}.} slots from the beginning
of the next epoch and one where \var{s} is less than \RandomnessStabilisationWindow{}
slots until the start of the next epoch.

Note that in \ref{eq:update-both}, the candidate nonce $\eta_c$ transitions to
$\eta_v\nonceOp\eta$, not $\eta_c\nonceOp\eta$. The reason for this is that even
though the candidate nonce is frozen sometime during the epoch, we want the two
nonces to again be equal at the start of a new epoch.

\begin{figure}[ht]

  \begin{equation}\label{eq:update-both}
    \inference[Update-Both]
    {
      s < \fun{firstSlot}~((\epoch{s}) + 1) - \RandomnessStabilisationWindow
    }
    {
      {\begin{array}{c}
         \eta \\
       \end{array}}
      \vdash
      {\left(\begin{array}{c}
            \eta_v \\
            \eta_c \\
      \end{array}\right)}
      \trans{updn}{\var{s}}
      {\left(\begin{array}{c}
            \varUpdate{\eta_v\nonceOp\eta} \\
            \varUpdate{\eta_v\nonceOp\eta} \\
      \end{array}\right)}
    }
  \end{equation}

  \nextdef

  \begin{equation}\label{eq:only-evolve}
    \inference[Only-Evolve]
    {
      s \geq \fun{firstSlot}~((\epoch{s}) + 1) - \RandomnessStabilisationWindow
    }
    {
      {\begin{array}{c}
         \eta \\
       \end{array}}
      \vdash
      {\left(\begin{array}{c}
            \eta_v \\
            \eta_c \\
      \end{array}\right)}
      \trans{updn}{\var{s}}
      {\left(\begin{array}{c}
            \varUpdate{\eta_v\nonceOp\eta} \\
            \eta_c \\
      \end{array}\right)}
    }
  \end{equation}
  \caption{Update Nonce rules}
  \label{fig:rules:update-nonce}
\end{figure}

\subsection{$\mathsf{OCERT}$ Transition}
\label{sec:oper-cert-trans}

The Operational Certificate Transition ($\mathsf{OCERT}$) environment consists of the set of stake
pools $\var{stpools}$. Its state is the mapping of operation certificate issue numbers.
Its signal is a block header.

\begin{figure}
  \emph{Operational Certificate Transitions}
  \begin{equation*}
    \var{\_} \vdash \var{\_} \trans{ocert}{\_} \var{\_} \subseteq
    \powerset (\powerset{\type{KeyHash}} \times \KeyHash_{pool} \mapsto \N \times \BHeader \times \KeyHash_{pool} \mapsto \N)
  \end{equation*}
  %
  \emph{Operational Certificate helper function}
  \begin{align*}
      & \fun{currentIssueNo} \in \powerset{\type{KeyHash}} \to (\KeyHash_{pool} \mapsto \N)
                                           \to \KeyHash_{pool}
                                           \to \N^? \\
      & \fun{currentIssueNo}~\var{stpools}~ \var{cs} ~\var{hk} =
      \begin{cases}
        \var{hk}\mapsto \var{n} \in \var{cs} & n \\
        \var{hk} \in \var{stpools} & 0 \\
        \text{otherwise} & \Nothing \; (\ref{itm:ocert-failures-7})
      \end{cases}
  \end{align*}
  \caption{Operational Certificate transition-system types}
  \label{fig:ts-types:ocert}
\end{figure}

The transition rule $\mathsf{OCERT}$ is shown in Figure~\ref{fig:rules:ocert}. From the block
header body \var{bhb} we first extract the following:

\begin{itemize}
  \item The operational certificate, consisting of the hot key \var{vk_{hot}},
    the certificate issue number \var{n}, the KES period start \var{c_0} and the cold key
  signature $\tau$.
\item The cold key \var{vk_{cold}}.
\item The slot \var{s} for the block.
\item The number of KES periods that have elapsed since the start period on the certificate.
\end{itemize}

Using this we verify the preconditions of the operational certificate state
transition which are the following:

\begin{itemize}
\item The KES period of the slot in the block header body must be greater than or equal to
  the start value \var{c_0} listed in the operational certificate,
  and less than $\MaxKESEvo$-many KES periods after \var{c_0}.
  The value of $\MaxKESEvo$ is the agreed-upon lifetime of an operational certificate,
  see \cite{delegation_design}.
\item \var{hk} exists as key in the mapping of certificate issues numbers to a KES
  period \var{m} and that period is less than or equal to \var{n}. Also, \var{n} must
  be less than or equal to the successor of \var{m}.
\item The signature $\tau$ can be verified with the cold verification key
  \var{vk_{cold}}.
\item The KES signature $\sigma$ can be verified with the hot verification key
  \var{vk_{hot}}.
\end{itemize}

After this, the transition system updates the operational certificate state by
updating the mapping of operational certificates where it overwrites the entry
of the key \var{hk} with the KES period \var{n}.

\begin{figure}[ht]
  \begin{equation}\label{eq:ocert}
    \inference[OCert]
    {
      (\var{bhb},~\sigma)\leteq\var{bh}
      &
      (\var{vk_{hot}},~n,~c_{0},~\tau) \leteq \hBOCert{bhb}
      &
      \var{vk_{cold}} \leteq \hBVkCold{bhb}
      \\
      \var{hk} \leteq \hashKey{vk_{cold}}
      &
      \var{s}\leteq\hBSlot{bhb}
      &
      t \leteq \kesPeriod{s} - c_0
      \\~\\
      c_0 \overset{(\ref{itm:ocert-failures-1})}{\leq} \kesPeriod{s} \overset{(\ref{itm:ocert-failures-2})}{<} c_0 + \MaxKESEvo
      \\
      \fun{currentIssueNo} ~ \var{stpools} ~ \var{cs} ~ \var{hk} = m
      &
      n \in \{\overset{(\ref{itm:ocert-failures-3})}{m}, \overset{(\ref{itm:ocert-failures-4})}{m + 1}\}
      \\~\\
      \mathcal{V}_{\var{vk_{cold}}}{\serialised{(\var{vk_{hot}},~n,~c_0)}}_{\tau} \; (\ref{itm:ocert-failures-5})
      &
      \mathcal{V}^{\mathsf{KES}}_{vk_{hot}}{\serialised{bhb}}_{\sigma}^{t} \; (\ref{itm:ocert-failures-6})
      \\
    }
    {
      \var{stpools}\vdash\var{cs}
      \trans{ocert}{\var{bh}}\varUpdate{\var{cs}\unionoverrideRight\{\var{hk}\mapsto n\}}
    }
  \end{equation}
  \caption{Operational Certificate rules}
  \label{fig:rules:ocert}
\end{figure}

The $\mathsf{OCERT}$ rule has the following predicate failures:
\begin{enumerate}
\item \label{itm:ocert-failures-1} If the KES period is less than the KES period start in the certificate,
  there is a \emph{KESBeforeStart} failure.
\item \label{itm:ocert-failures-2} If the KES period is greater than or equal to the KES period end (start +
  $\MaxKESEvo$) in the certificate, there is a \emph{KESAfterEnd} failure.
\item \label{itm:ocert-failures-3} If the period counter in the original key hash counter mapping is larger
  than the period number in the certificate, there is a \emph{CounterTooSmall}
  failure.
\item \label{itm:ocert-failures-4} If the period number in the certificate is larger than the successor of
  the period counter in the original key hash counter mapping, there is a
  \emph{CounterOverIncremented} failure.
\item \label{itm:ocert-failures-5} If the signature of the hot key, KES period number and period start is
  incorrect, there is an \emph{InvalidSignature} failure.
\item \label{itm:ocert-failures-6} If the KES signature using the hot key of the block header body is
  incorrect, there is an \emph{InvalideKesSignature} failure.
\item \label{itm:ocert-failures-7} If there is no entry in the key hash to counter mapping for the cold key,
  there is a \emph{NoCounterForKeyHash} failure.
\end{enumerate}

\subsection{Verifiable Random Function}
\label{sec:verif-rand-funct}

In this section we define a function $\fun{praosVrfChecks}$ which performs all the VRF related checks
on a given block header body.
In addition to the block header body, the function requires the epoch nonce,
the stake distribution (aggregated by pool), and the active slots coefficient from the protocol
parameters. The function checks:

\begin{itemize}
\item The validity of the proofs for the leader value and the new nonce.
\item The verification key is associated with relative stake $\sigma$ in the stake distribution.
\item The $\fun{hBLeader}$ value of \var{bhb} indicates a possible leader for
  this slot. The function $\fun{checkLeaderVal}$ is defined in \ref{sec:leader-value-calc}.
\end{itemize}

\begin{figure}
  \begin{align*}
      & \fun{issuerIDfromBHBody} \in \BHBody \to \KeyHash_{pool} \\
      & \fun{issuerIDfromBHBody} = \hashKey{} \circ \hBVkCold{} \\
  \end{align*}
  %
  \begin{align*}
      & \fun{vrfChecks} \in \Nonce \to \BHBody \to \Bool \\
      & \fun{vrfChecks}~\eta_0~\var{bhb} =
          \verifyVrf{\VRFRes}{\var{vrfK}}{((\slotToSeed{slot})~\XOR~(\nonceToSeed{\eta_0}))}{(\var{proof},~\var{value}}) \\
      & ~~~~\where \\
      & ~~~~~~~~~~\var{slot} \leteq \hBSlot{bhb} \\
      & ~~~~~~~~~~\var{vrfK} \leteq \fun{hBVrfVk}~\var{bhb} \\
      & ~~~~~~~~~~\var{value} \leteq \fun{hBVrfRes}~\var{bhb} \\
      & ~~~~~~~~~~\var{proof} \leteq \fun{hBVrfPrf}~\var{bhb} \\
  \end{align*}
  %
  \begin{align*}
      & \fun{praosVrfChecks} \in \Nonce \to \PoolDistr \to \unitIntervalNonNull \to \BHBody \to \Bool \\
      & \fun{praosVrfChecks}~\eta_0~\var{pd}~\var{f}~\var{bhb} = \\
      & \begin{array}{cl}
        ~~~~ & \var{hk}\mapsto (\sigma,~\var{vrfHK})\in\var{pd} \; (\ref{itm:vrf-failures-1}) \\
        ~~~~ \land & \var{vrfHK} = \hashKey{vrfK} \; (\ref{itm:vrf-failures-2}) \\
        ~~~~ \land & \fun{vrfChecks}~\eta_0~\var{bhb} \; (\ref{itm:vrf-failures-3}) \\
        ~~~~ \land & \fun{checkLeaderVal}~(\fun{hBLeader}~\var{bhb})~\sigma~\var{f} \; (\ref{itm:vrf-failures-4}) \\
      \end{array} \\
      & ~~~~\where \\
      & ~~~~~~~~~~\var{hk} \leteq \fun{issuerIDfromBHBody}~\var{bhb} \\
      & ~~~~~~~~~~\var{vrfK} \leteq \fun{hBVrfVk}~\var{bhb} \\
  \end{align*}
  \caption{VRF helper functions}
  \label{fig:vrf-checks}
\end{figure}

The definition of $\fun{praosVrfChecks}$ is shown in Figure~\ref{fig:vrf-checks}
and has the following predicate failures:

\begin{enumerate}
\item \label{itm:vrf-failures-1} If the VRF key is not in the pool distribution, there is a
  \emph{VRFKeyUnknown} failure.
\item \label{itm:vrf-failures-2} If the VRF key hash does not match the one listed in the block header,
  there is a \emph{VRFKeyWrongVRFKey} failure.
\item \label{itm:vrf-failures-3} If the VRF generated value in the block header does not validate
  against the VRF certificate, there is a \emph{VRFKeyBadProof} failure.
\item \label{itm:vrf-failures-4} If the VRF generated leader value in the block header is too large
  compared to the relative stake of the pool, there is a \emph{VRFLeaderValueTooBig} failure.
\end{enumerate}

\clearpage

\subsection{$\mathsf{PRTCL}$ Transition}
\label{sec:protocol-trans}

The Protocol Transition ($\mathsf{PRTCL}$) calls the transition $\mathsf{UPDN}$ to update the evolving
and candidate nonces, and checks the operational certificate with $\mathsf{OCERT}$.

\begin{figure}
  \emph{Protocol environments}
  \begin{equation*}
    \PrtclEnv =
    \left(
      \begin{array}{r@{~\in~}lr}
        \var{pd} & \PoolDistr & \text{pool stake distribution} \\
        \eta_0 & \Nonce & \text{epoch nonce} \\
      \end{array}
    \right)
  \end{equation*}
  %
  \emph{Protocol states}
  \begin{equation*}
    \PrtclState =
    \left(
      \begin{array}{r@{~\in~}lr}
        \var{cs} & \KeyHash_{pool} \mapsto \N & \text{operational certificate issues numbers} \\
        \eta_v & \Nonce & \text{evolving nonce} \\
        \eta_c & \Nonce & \text{candidate nonce} \\
      \end{array}
    \right)
  \end{equation*}
  %
  \emph{Protocol Transitions}
  \begin{equation*}
    \_ \vdash \var{\_} \trans{prtcl}{\_} \var{\_} \subseteq
    \powerset (\PrtclEnv \times \PrtclState \times \BHeader \times \PrtclState)
  \end{equation*}
  \caption{Protocol transition-system types}
  \label{fig:ts-types:prtcl}
\end{figure}

The environments for this transition are:
\begin{itemize}
  \item The stake pool stake distribution $\var{pd}$.
  \item The epoch nonce $\eta_0$.
\end{itemize}

The states for this transition consists of:
\begin{itemize}
  \item The operational certificate issue number mapping.
  \item The evolving nonce.
  \item The candidate nonce for the next epoch.
\end{itemize}

\begin{figure}[ht]
  \begin{equation}\label{eq:prtcl}
    \inference[PRTCL]
    {
      \var{bhb}\leteq\headerBody{bh} &
      \eta\leteq\fun{hBNonce}~bhb &
      slot\leteq\fun{hBSlot}~bhb
      \\~\\
      {
        \eta
        \vdash
        {\left(\begin{array}{c}
        \eta_v \\
        \eta_c \\
        \end{array}\right)}
        \trans{\hyperref[fig:rules:update-nonce]{updn}}{\var{slot}}
        {\left(\begin{array}{c}
        \eta_v' \\
        \eta_c' \\
        \end{array}\right)}
    }\\~\\
      {
        \dom{\var{pd}}\vdash\var{cs}\trans{\hyperref[fig:rules:ocert]{ocert}}{\var{bh}}\var{cs'}
      }
      \\~\\
      \fun{praosVrfChecks}~\eta_0~\var{pd}~\ActiveSlotCoeff~\var{bhb}
    }
    {
      {\begin{array}{c}
         \var{pd} \\
         \eta_0 \\
       \end{array}}
      \vdash
      {\left(\begin{array}{c}
            \var{cs} \\
            \eta_v \\
            \eta_c \\
      \end{array}\right)}
      \trans{prtcl}{\var{bh}}
      {\left(\begin{array}{c}
            \varUpdate{cs'} \\
            \varUpdate{\eta_v'} \\
            \varUpdate{\eta_c'} \\
      \end{array}\right)}
    }
  \end{equation}
  \caption{Protocol rules}
  \label{fig:rules:prtcl}
\end{figure}

This transition establishes that a block producer is in fact authorized.
Since there are three key pairs involved (cold keys, VRF keys, and hot KES keys)
it is worth examining the interaction closely.

\begin{itemize}
  \item First we check the operational certificate with $\mathsf{OCERT}$.
  This uses the cold verification key given in the block header.
  We do not yet trust that this key is a registered pool key.
  If this transition is successful, we know that the cold key in the block header has authorized
  the block.
\item  Next, in the $\fun{vrfChecks}$ predicate, we check that the hash of this cold key is in the
  mapping $\var{pd}$, and that it maps to $(\sigma,~\var{hk_{vrf}})$, where
  $(\sigma,~\var{hk_{vrf}})$ is the hash of the VRF key in the header.
  If $\fun{praosVrfChecks}$ returns true, then we know that the cold key in the block header was a
  registered stake pool at the beginning of the previous epoch, and that it is indeed registered
  with the VRF key listed in the header.
\item Finally, we use the VRF verification key in the header, along with the VRF proofs in the
  header, to check that the operator is allowed to produce the block.
\end{itemize}

\clearpage

\subsection{$\mathsf{TICKF}$ Transition}
\label{sec:tickf-trans}

The Tick Forecast Transition ($\mathsf{TICKF}$) performs some chain level
upkeep. The state is the epoch specific state necessary for the $\mathsf{NEWEPOCH}$ transition.

Part of the upkeep is updating the genesis key delegation mapping
according to the future delegation mapping.
For each genesis key, we adopt the most recent delegation in $\var{fGenDelegs}$
that is past the current slot, and any future genesis key delegations past the current
slot is removed. The helper function $\fun{adoptGenesisDelegs}$ accomplishes the update.

\begin{figure}
  \emph{Tick Forecast Transitions}
  \begin{equation*}
    \vdash \var{\_} \trans{tickf}{\_} \var{\_} \subseteq
    \powerset (\NewEpochState \times \Slot \times \NewEpochState)
  \end{equation*}  %
  \emph{Tick Forecast helper function}
  \begin{align*}
      & \fun{adoptGenesisDelegs} \in \EpochState \to \Slot \to \EpochState
      \\
      & \fun{adoptGenesisDelegs}~\var{es}~\var{slot} = \var{es'}
      \\
      & ~~~~\where
      \\
      & ~~~~~~~~~~
      (\var{acnt},~\var{ss},(\var{us},(\var{ds},\var{ps})),~\var{prevPp},~\var{pp})
      \leteq\var{es}
      \\
      & ~~~~~~~~~~
      (~\var{rewards},~\var{delegations},~\var{ptrs},
      ~\var{fGenDelegs},~\var{genDelegs},~\var{i_{rwd}})\leteq\var{ds}
      \\
      & ~~~~~~~~~~\var{curr}\leteq
        \{
          (\var{s},~\var{gkh})\mapsto(\var{vkh},~\var{vrf})\in\var{fGenDelegs}
          ~\mid~
          \var{s}\leq\var{slot}
        \}
      \\
      & ~~~~~~~~~~\var{fGenDelegs'}\leteq
          \var{fGenDelegs}\setminus\var{curr}
      \\
      & ~~~~~~~~~~\var{genDelegs'}\leteq
          \left\{
            \var{gkh}\mapsto(\var{vkh},~\var{vrf})
            ~\mathrel{\Bigg|}~
            {
              \begin{array}{l}
                (\var{s},~\var{gkh})\mapsto(\var{vkh},~\var{vrf})\in\var{curr}\\
                \var{s}=\max\{s'~\mid~(s',~\var{gkh})\in\dom{\var{curr}}\}
              \end{array}
            }
          \right\}
      \\
      & ~~~~~~~~~~\var{ds'}\leteq
          (\var{rewards},~\var{delegations},~\var{ptrs},
          ~\var{fGenDelegs'},~\var{genDelegs}\unionoverrideRight\var{genDelegs'},~\var{i_{rwd}})
      \\
      & ~~~~~~~~~~\var{es'}\leteq
      (\var{acnt},~\var{ss},(\var{us},(\var{ds'},\var{ps})),~\var{prevPp},~\var{pp})
  \end{align*}
  \caption{Tick Forecast transition-system types}
  \label{fig:ts-types:tickf}
\end{figure}

The $\mathsf{TICKF}$ transition rule is shown in Figure~\ref{fig:rules:tickf}.
The signal is a slot \var{s}.

One sub-transition is done:
The $\mathsf{NEWEPOCH}$ transition performs any state change needed if it is the first
block of a new epoch.

\begin{figure}[ht]
  \begin{equation}\label{eq:tickf}
    \inference[TickForecast]
    {
      {
        \vdash
        \var{nes}
        \trans{\hyperref[fig:ts-types:newepoch]{newepoch}}{\epoch{s}}
        \var{nes}'
      }
      \\~\\
      (\var{e_\ell'},~\var{b_{prev}'},~\var{b_{cur}'},~\var{es'},~\var{ru'},~\var{pd'})
      \leteq\var{nes'}
      \\
      \var{es''}\leteq\fun{adoptGenesisDelegs}~\var{es'}~\var{s}
      \\
      \var{forecast}\leteq
      (\var{e_\ell'},~\var{b_{prev}'},~\var{b_{cur}'},~\var{es''},~\var{ru'},~\var{pd'})
      \\~\\
    }
    {
      \vdash\var{nes}\trans{tickf}{\var{s}}\varUpdate{\var{forecast}}
    }
  \end{equation}
  \caption{Tick Forecast rules}
  \label{fig:rules:tickf}
\end{figure}

\clearpage

\subsection{$\mathsf{CHAINHEAD}$ Transition}
\label{sec:chainhead-trans}

The Chain Head Transition rule ($\mathsf{CHAINHEAD}$) is the main rule of the blockchain layer
part of the STS. It calls $\mathsf{TICKF}$, $\mathsf{TICKN}$, and $\mathsf{PRTCL}$,
as sub-rules.

The transition checks the following things:
(via $\fun{chainChecks}$ and $\fun{prtlSeqChecks}$ from Figure~\ref{fig:funcs:chainhead-helper}):
\begin{itemize}
\item The slot in the block header body is larger than the last slot recorded.
\item The block number increases by exactly one.
\item The previous hash listed in the block header matches the previous
  block header hash which was recorded.
\item The size of the block header is less than or equal to the maximal size that the
  protocol parameters allow for block headers.
\item The size of the block body, as claimed by the block header, is less than or equal to the
  maximal size that the protocol parameters allow for block bodies.
\item The node is not obsolete, meaning that the major component of the
  protocol version in the protocol parameters
  is not bigger than the constant $\MaxMajorPV$.
\end{itemize}


The $\mathsf{CHAINHEAD}$ state is shown in Figure~\ref{fig:ts-types:chainhead}, it consists of the
following:

\begin{itemize}
  \item The operational certificate issue number map $\var{cs}$.
  \item The epoch nonce $\eta_0$.
  \item The evolving nonce $\eta_v$.
  \item The candidate nonce $\eta_c$.
  \item The previous epoch hash nonce $\eta_h$.
  \item The last header hash \var{h}.
  \item The last slot \var{s_\ell}.
  \item The last block number \var{b_\ell}.
\end{itemize}

\begin{figure}
  \emph{Chain Head states}
  \begin{equation*}
    \LastAppliedBlock =
    \left(
      \begin{array}{r@{~\in~}lr}
        \var{b_\ell} & \Slot & \text{last block number} \\
        \var{s_\ell} & \Slot & \text{last slot} \\
        \var{h} & \HashHeader & \text{latest header hash} \\
      \end{array}
    \right)
  \end{equation*}
  %
  \begin{equation*}
    \ChainHeadState =
    \left(
      \begin{array}{r@{~\in~}lr}
        \var{cs} & \KeyHash_{pool} \mapsto \N & \text{operational certificate issue numbers} \\
        ~\eta_0 & \Nonce & \text{epoch nonce} \\
        ~\eta_v & \Nonce & \text{evolving nonce} \\
        ~\eta_c & \Nonce & \text{candidate nonce} \\
        ~\eta_h & \Nonce & \text{nonce from hash of last epoch's last header} \\
        \var{lab} & \LastAppliedBlock^? & \text{latest applied block} \\
      \end{array}
    \right)
  \end{equation*}
  %
  \emph{Chain Head Transitions}
  \begin{equation*}
    \_ \vdash \var{\_} \trans{chainhead}{\_} \var{\_} \subseteq
    \powerset (\NewEpochState \times \ChainHeadState \times \BHeader \times \ChainHeadState)
  \end{equation*}
  \caption{Chain Head transition-system types}
  \label{fig:ts-types:chainhead}
\end{figure}

The $\mathsf{CHAINHEAD}$ transition rule is shown in Figure~\ref{fig:rules:chainhead}.
It contains a new epoch state $\var{nes}$ in the environment and its signal is a block
header $\var{bh}$.
The transition uses a few helper functions defined in Figure~\ref{fig:funcs:chainhead-helper}.

%%
%% Figure - Helper Functions for Chain Head Rules
%%
\begin{figure}[htb]
  \emph{Chain Head Transition Helper Functions}
  \begin{align*}
      & \fun{chainChecks} \in \N \to (\N,\N, \ProtVer) \to \BHeader \to \Bool \\
      & \fun{chainChecks}~\var{maxpv}~(\var{maxBHSize},~\var{maxBBSize},~\var{protocolVersion})~\var{bh} = \\
      & ~~~~ m \overset{(\ref{itm:chainhead-failures-6})}{\leq} \var{maxpv} \\
      & ~~~~ \land~\headerSize{bh} \overset{(\ref{itm:chainhead-failures-4})}{\leq} \var{maxBHSize} \\
      & ~~~~ \land~\hBBodySize{(\headerBody{bh})} \overset{(\ref{itm:chainhead-failures-5})}{\leq} \var{maxBBSize} \\
      & ~~~~ \where (m,~\wcard)\leteq\var{protocolVersion}
  \end{align*}
  %
  \begin{align*}
      & \fun{lastAppliedHash} \in \LastAppliedBlock^? \to \HashHeader^? \\
      & \fun{lastAppliedHash}~\var{lab} =
        \begin{cases}
          \Nothing & lab = \Nothing \\
          h & lab = (\wcard,~\wcard,~h) \\
        \end{cases}
  \end{align*}
  %
  \begin{align*}
      & \fun{prtlSeqChecks} \to \LastAppliedBlock^? \to \BHeader \to \Bool \\
      & \fun{prtlSeqChecks}~\var{lab}~\var{bh} =
        \begin{cases}
          \mathsf{True}
          &
          lab = \Nothing
          \\
          \var{s_\ell} \overset{(\ref{itm:chainhead-failures-1})}{<} \var{slot}
          \land \var{b_\ell} + 1 \overset{(\ref{itm:chainhead-failures-2})}{=} \var{bn}
          \land \var{ph} \overset{(\ref{itm:chainhead-failures-3})}{=} \hBPrevHeader{bhb}
          &
          lab = (b_\ell,~s_\ell,~\wcard) \\
        \end{cases} \\
      & ~~~~\where \\
      & ~~~~~~~~~~\var{bhb} \leteq \headerBody{bh} \\
      & ~~~~~~~~~~\var{bn} \leteq \hBBlockNo{bhb} \\
      & ~~~~~~~~~~\var{slot} \leteq \hBSlot{bhb} \\
      & ~~~~~~~~~~\var{ph} \leteq \lastAppliedHash{lab} \\
  \end{align*}

  \caption{Helper functions used in the Chain Head transition}
  \label{fig:funcs:chainhead-helper}
\end{figure}

\begin{figure}[ht]
  \begin{equation}\label{eq:chain-head}
    \inference[ChainHead]
    {
      \var{bhb} \leteq \headerBody{bh}
      &
      \var{s} \leteq \hBSlot{bhb}
      \\~\\
      \fun{prtlSeqChecks}~\var{lab}~\var{bh}
      \\~\\
      {
        \vdash\var{nes}\trans{\hyperref[fig:rules:tickf]{tickf}}{\var{s}}\var{forecast}
      } \\~\\
      (\var{e_1},~\wcard,~\wcard,~\wcard,~\wcard,~\wcard)
        \leteq\var{nes} \\
      (\var{e_2},~\wcard,~\wcard,~\var{es},~\wcard,~\var{pd})
        \leteq\var{forecast} \\
        (\wcard,~\wcard,~\wcard,~\wcard,~\var{pp})\leteq\var{es}\\
          \var{ne} \leteq  \var{e_1} \neq \var{e_2}\\
          \eta_{ph} \leteq \prevHashToNonce{(\lastAppliedHash{lab})} \\~\\
      \fun{chainChecks}~
        \MaxMajorPV~(\fun{maxHeaderSize}~\var{pp},~\fun{maxBlockSize}~\var{pp},~\fun{pv}~\var{pp})~
        \var{bh}\\~\\
      {
        {\begin{array}{c}
        \eta_c \\
        \eta_\var{ph} \\
        \end{array}}
        \vdash
        {\left(\begin{array}{c}
        \eta_0 \\
        \eta_h \\
        \end{array}\right)}
        \trans{\hyperref[fig:rules:tick-nonce]{tickn}}{\var{ne}}
        {\left(\begin{array}{c}
        \eta_0' \\
        \eta_h' \\
        \end{array}\right)}
      }\\~\\~\\
      {
        {\begin{array}{c}
            \var{pd} \\
            \eta_0' \\
         \end{array}}
        \vdash
        {\left(\begin{array}{c}
              \var{cs} \\
              \eta_v \\
              \eta_c \\
        \end{array}\right)}
        \trans{\hyperref[fig:rules:prtcl]{prtcl}}{\var{bh}}
        {\left(\begin{array}{c}
              \var{cs'} \\
              \eta_v' \\
              \eta_c' \\
        \end{array}\right)}
      } \\~\\~\\
      \var{lab'}\leteq (\hBBlockNo{bhb},~\var{s},~\hBBodyHash{bhb} ) \\
    }
    {
      \var{nes}
      \vdash
      {\left(\begin{array}{c}
            \var{cs} \\
            \eta_0 \\
            \eta_v \\
            \eta_c \\
            \eta_h \\
            \var{lab} \\
      \end{array}\right)}
      \trans{chainhead}{\var{bh}}
      {\left(\begin{array}{c}
            \var{cs}' \\
            \eta_0' \\
            \eta_v' \\
            \eta_c' \\
            \eta_h' \\
            \var{lab}' \\
      \end{array}\right)}
    }
  \end{equation}
  \caption{Chain Head rules}
  \label{fig:rules:chainhead}
\end{figure}

The $\mathsf{CHAINHEAD}$ transition rule has the following predicate failures:
\begin{enumerate}
\item \label{itm:chainhead-failures-1} If the slot of the block header body is not larger than the last slot,
  there is a \emph{WrongSlotInterval} failure.
\item \label{itm:chainhead-failures-2} If the block number does not increase by exactly one,
  there is a \emph{WrongBlockNo} failure.
\item \label{itm:chainhead-failures-3} If the hash of the previous header of the block header body is not equal
  to the last header hash, there is a \emph{WrongBlockSequence} failure.
\item \label{itm:chainhead-failures-4} If the size of the block header is larger than the maximally allowed size,
  there is a \emph{HeaderSizeTooLarge} failure.
\item \label{itm:chainhead-failures-5} If the size of the block body is larger than the maximally allowed size,
  there is a \emph{BlockSizeTooLarge} failure.
\item \label{itm:chainhead-failures-6} If the major component of the protocol version is larger than $\MaxMajorPV$,
  there is a \emph{ObsoleteNode} failure.
\end{enumerate}

%%% Local Variables:
%%% mode: latex
%%% TeX-master: "ledger-spec"
%%% End:
